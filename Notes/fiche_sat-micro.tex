\documentclass[a4paper, 10pt]{article}

\usepackage[utf8]{inputenc}
\usepackage[T1]{fontenc}
\usepackage[frenchb]{babel}

\usepackage{bussproofs}

\usepackage{multicol}
\usepackage{listings}
\usepackage{graphicx}
\usepackage{hyperref}
\usepackage{amssymb}
\usepackage{amsmath}
\usepackage{syntax}

\title{Analyse de l'article \emph{SAT-Micro : petit mais costaud !} de
  S. Conchon et al.}
\author{B. Carré}

\begin{document}
\maketitle


\section*{introduction}
Le problème de satisfaisabilité d'une formule propositionnelle est un
problème NP-complet (parcours de la table de vérité).
Le \emph{Sat-micro} (un \emph{Sat-solver}), qui décide de la
satisfaisabilité de ces formules de \emph{forme normale conjonctive}
(FNC), est basé sur un algorithme \emph{DPLL}, avec quelques
optimisations telles que le retour en arrière chronologique ou
l'apprentissage.

\section{L'algorithme DPLL}
\subsection{L'algorithme}
Une FNC est de la forme $\bigwedge^n_{i=1}(l_1\vee\ldots\vee l_{k_i})$
avec $l_j$ un \emph{littéral} et un ensemble de disjonction est appelé \emph{clause}.

L'algorithme essaie d'attribuer des valeurs au variables, qui rendent
la formule vraie. Mais avec deux règles particulières qui font qu'il
n'y a pas besoin de vérifier toutes les possibilités de la table de
vérité.
\begin{itemize}
\item La \emph{propagation par contrainte} permet, lorsqu'on attribue
  une valeur à un littéral, de supprimer les littéraux devenus faux des
  clauses, et de supprimer toute une clause dont un des littéraux est
  devenu vrai (car la clause est vraie quelque soient les valeurs des
  autres littéraux) .
\item la \emph{clause unitaire} nous indique que s'il n'y a qu'un seul
  littéral dans uen clause, alors ce littéral est défini vrai, pour que
  la clause soit vraie.
\end{itemize}
DPLL attribue ainsi des valeurs aux variables, avec deux
conditions d'arrêt :
\begin{itemize}
\item si on obtient la formule vide, alors on a trouvé une
  instanciation pour chaque variable qui rend la formule satisfaisable.
\item si on obtient une clause vide dans la formule, alors
  l'instanciation rend la clause (et donc la formule)
  non-satisfaisable (ce qu'on appelle un \emph{conflit}), il faut
  alors revenir en arrière pour essayer de trouver une instanciation
  différente.  
\end{itemize}
Les règles d'inférences pour DPLL sont définies ainsi :

%% \begin{multicols}{2}
\begin{prooftree}
  \def\fCenter{\mbox{\ $\vdash$\ }}
  \AxiomC{}
  \LeftLabel{CONFLICT}
  \UnaryInf$\Gamma \fCenter \Delta, \emptyset$
\end{prooftree}

\emph{CONFLICT} : si on obtiens une clause vide, il y a un conflit, il
faut alors retourner en arrière pour chercher d'autres instanciacions.

\begin{prooftree}
  \def\fCenter{\mbox{\ $\vdash$\ }}
  \AxiomC{$\Gamma,l \vdash \Delta$}
  \LeftLabel{ASSUME}
  \UnaryInf$\Gamma \fCenter \Delta, l$
\end{prooftree}

\emph{ASSUME} : si on a une clause contenant qu'un seul litéral $l$,
alors il doit être supposé vrai (C'est la \emph{clause unitaire}).

\begin{prooftree}
  \def\fCenter{\mbox{\ $\vdash$\ }}
  \AxiomC{$\Gamma,l \vdash \Delta$}
  \AxiomC{$\Gamma,\neg l \vdash \Delta$}
  \LeftLabel{UNSAT}
  \BinaryInf$\Gamma \fCenter \Delta$
\end{prooftree}

\emph{UNSAT} : un litéral est d'abbord supposé vrai, et si aucune
instanciations satisfaisante n'est trouvée(i.e. si toutes le sbranches
de gauche termientn avec \emph{CONFLICT}), il est supposé faux.

\begin{prooftree}
  \def\fCenter{\mbox{\ $\vdash$\ }}
  \AxiomC{$\Gamma, l \vdash \Delta, C$}
  \LeftLabel{BCP_1}
  \UnaryInf$\Gamma, l \fCenter \Delta,\neg l \vee C$
\end{prooftree}

\emph{BCP_1} : première règle de propagation par contrainte : les
littéraux devenus faux dans les clauses sont supprimés.

\begin{prooftree}
  \def\fCenter{\mbox{\ $\vdash$\ }}
  \AxiomC{$\Gamma, l \vdash \Delta$}
  \LeftLabel{BCP_2}
  \UnaryInf$\Gamma, l \fCenter \Delta, l \vee C$
\end{prooftree}

\emph{BCP_2} : seconde règle de propagation par contrainte, si une
clause contient un littéral devenu vrai, elle est supprimé de la formule

%% \end{multicols}
\bigskip

Elles décrivent l'état de l'algorithme ainsi :
$\Gamma \vdash \Delta$ avec $\Gamma$ est l'ensemble des litéraux
supposés vrais, et $\Delta$ la formule courante, avec TODO

\subsection{L'implémentation}

La représention de la FNC est indépendante de l'implémentaiton, les
types qui représentent un littéral et une formule sont abstraits

Le module Sat, est paramétré par un module de type FNC, et contient un
module de type LITERAL, et un module de type Set.S pour l'ensemble des
litéraux.

Le type t du module contient un ensemble de litéraux \emph{gamma} et
la formule \emph{delta} sous forme FNC.

La fonciton \emph{assume} introduit un littéral dans l'environnement
puis fait appel à la fonction \emph{bcp}.
La fonction \emph{bcp} parcourt la formule et enlève tous les littéraux
dont leur négation est dans \emph{gamma}, et si la littéral est dans
gamma, toute la clause dans laquelle il est est supprimée.

TODO

\section{Améliorations}
\subsection{Retour en arrière non chronologique}
L'idée est d'éviter de parcourir des branches inutilement si on
s'appercoit que l'ajout d'un littéral avec la règle \emph{UNSAT} n'a
pas eu d'impacte sur la branche gauche, il est inutile d'aller
vérifier la branche droite.

Pour cela, on a besoin de garder en mémoire les dépendances qui ont
conduit à l'ajout d'un littéral, et lorsque l'algorithme tombe sur un
clonflit, on sait qu'il faut revenir en arrière à la dernière
dépendance liée au conflit, car il est inutile de parcourir les autres
branches qui n'ont pas d'impacts sur ce conflit.

Ona joute au contexte des annotations aux littéraux, qui correspondent
aux dépendances (les littéraux qui ont conduit à l'ajout de celui-ci).
On ajoute aussi des annotations aux clauses de la formules, qui
permettent de savoir avec quels littéraux celles-ci ont été réduites.

\begin{multicols}{2}
\begin{prooftree}
  \def\fCenter{\mbox{\ $\vdash$\ }}
  \AxiomC{}
  \LeftLabel{CONFLICT}
  \UnaryInf$\Gamma \fCenter \Delta, \emptyset[\mathcal{A}] : \mathcal{A}$
\end{prooftree}

\begin{prooftree}
  \def\fCenter{\mbox{\ $\vdash$\ }}
  \AxiomC{$\Gamma,l[\mathcal{B}] \vdash \Delta : \mathcal{A}$}
  \LeftLabel{ASSUME}
  \UnaryInf$\Gamma \fCenter \Delta, l[\mathcal{B}] : \mathcal{A}$
\end{prooftree}

\begin{prooftree}
  \def\fCenter{\mbox{\ $\vdash$\ }}
  \AxiomC{$\Gamma, l[\mathcal{B}] \vdash \Delta,C[\mathcal{B}\cup\mathcal{C}] : \mathcal{A}$}
  \LeftLabel{BCP_1}
  \UnaryInf$\Gamma, l[\mathcal{B}] \fCenter \Delta,\neg l \vee
  C[\mathcal{C}] : \mathcal{A}$
\end{prooftree}

\begin{prooftree}
  \def\fCenter{\mbox{\ $\vdash$\ }}
  \AxiomC{$\Gamma, l[\mathcal{B}] \vdash \Delta : \mathcal{A}$}
  \LeftLabel{BCP_2}
  \UnaryInf$\Gamma , l[\mathcal{B}] \fCenter \Delta, l \vee
  C[\mathcal{C}] : \mathcal{A}$
\end{prooftree}

\begin{prooftree}
  \def\fCenter{\mbox{\ $\vdash$\ }}
  \AxiomC{$\Gamma, l[l] \vdash \Delta : \mathcal{A}$}
  \LeftLabel{BJ}
  \UnaryInf$\Gamma \fCenter \Delta : \mathcal{A}$
\end{prooftree}
\end{multicols}

\begin{prooftree}
  \def\fCenter{\mbox{\ $\vdash$\ }}
  \AxiomC{$\Gamma,l[l] \vdash \Delta : \mathcal{A}$}
  \AxiomC{$\Gamma,\neg l[\mathcal{A}\backslash l] \vdash \Delta :
    \mathcal{B}$}
  \LeftLabel{UNSAT} 
  \RightLabel{$l \in \mathcal{A}$}
  \BinaryInf$\Gamma \fCenter \Delta : \mathcal{B}$
\end{prooftree}

\subsection{Apprentissage}


\section{Formes Normales Conjonctives équi-satisfaisables}

\section*{Conclusion}

\end{document}
