\documentclass[a4paper, 10pt]{report}

\usepackage[utf8]{inputenc}
\usepackage[frenchb]{babel}
\usepackage[T1]{fontenc}

\usepackage{multicol}
\usepackage{listings}
\usepackage{graphicx}
\usepackage{hyperref}
\usepackage{amssymb}
\usepackage{amsmath}
\usepackage{syntax}

\title{Analyse de l'article \emph{SAT-Micro : petit mais costaud !} de
  S. Conchon et al.}
\author{B. Carré}

\begin{document}
\maketitle


\section*{introduction}
Le problème de satisfaisabilité d'une formule propositionnelle est un
problème NP-complet (parcours de la table de vérité).
Le \emph{Sat-micro} (un \emph{Sat-solver}), qui décide de la
satisfaisabilité de ces formules de \emph{forme normale conjonctive}
(FNC), est basé sur un algorithme \emph{DPLL}, avec quelques
optimisations telles que le retour en arrière chronologique ou
l'apprentissage.

\section{L'algorithme DPLL}
Une FNC est de la forme $\bigwedge^n_{i=1}(l_1\vee\ldots\vee l_{k_i})$
avec $l_j$ un \emph{litéral} et un ensemble de disjonction est appelé \emph{clause}.

L'algorithme essaie d'attribuer des valeurs au variables, qui rendent
la formule vraie. Mais avec deux règles particulières qui font qu'il
n'y a pas besoin de vérifier toutes les possibilités de la table de vérité.

\section{Améliorations}
\subsection{Retour en arrière non chronologique}


\subsection{Apprentissage}

\section{Formes Normales Conjonctives équi-satisfaisables}

\section*{Conclusion}

\end{document}
